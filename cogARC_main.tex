% BSP for Bachelor in Game Programming
% english instead of norsk
% oneside for delivery, twoside for book style
\documentclass[BSP,english,oneside]{classes/gucthesis}

% For utf8 encoded .tex files
\usepackage[utf8]{inputenc}

% For colouring text
\usepackage[usenames,dvipsnames]{color}

% For cross references in pdf
\usepackage[pdftex]{graphicx,hyperref}
\hypersetup{
	colorlinks=true,
	linkcolor=Mahogany,
	urlcolor=RoyalBlue
}

% For appendix title
\usepackage{appendix}

% For footer on project frontpage
\usepackage{fancyhdr}

% For glossaries
\usepackage[nopostdot]{glossaries}
\usepackage{glossaries-babel}
\newglossaryentry{serialization}
{
	name=serialization,
	description={creates an internal representation of the visible data 
  				structure}
}

\newglossaryentry{ECMAScript}
{
	name=ECMAScript,
	description={Popularly known as Javascript, the most used scripting
				language throughout the web}
}

\newglossaryentry{Augmented Reality}
{
	name=Augmented Reality,
	description={A technology that enhance reality by combining digital and
				natural content, typically by combining camera, screen and
				processing power.}
}

\newglossaryentry{AR}
{
	name=AR,
	description={See \gls{Augmented Reality}}
}

\newglossaryentry{Vuforia}
{
	name=Vuforia,
	description={A free Augmented Reality software development kit that	offers
				vision-based image recognition for mobile platforms	and 
				Unity3D.}
}

\newglossaryentry{Inspector}
{
	name=Inspector,
	description={An area in the Unity Editor where you can manage editable
	properties of selected objects. In a script this can be public
	variables. In an object it can be position in space, scripts that are 
	attached, shaders that are used, etc.}
}

\newglossaryentry{Frame Marker}
{
	name=Frame Marker,
	description={A image recognized by the AR library as an object to be tracked and shown augments on.}
}

\newglossaryentry{Meta SpaceGlasses}
{
	name=Meta SpaceGlasses,
	description={Glasses that have screen capability on each eye. It features a camera and is therefore perfect for creating exiting experiences with \gls{Augmented Reality}}
}

\newglossaryentry{prefab}
{
	name=prefab,
	description={A premade object that can be added to the hiearchy in Unity, this object can have already added behavior as is the case with the camera used for our \gls{AR}.}
}

\makenoidxglossaries

% Remove '%' in front of \renewcommand the parent command

% Create command for commenting
\newcommand{\comment}[1]{\textcolor{blue}{\emph{#1}}}
%\renewcommand{\comment}[1]{}

% Create command for todo things
\newcommand{\todo}[1]{{\color{green}#1}}

% Norwegian Characters,  needs the {} or to be separate from the next letters
% \o{}   \aa{}   \ae{}   so at the end of a word you can use \o  \aa   \ae
% \O{}   \AA{}   \AE{}   you can also just leave a space and latex will remove it
%                        eg,  H\o gskolen i Gj\o vik


\begin{document}

\thesistitle{cogARC}
\thesisauthor{Per Kristian Warvik}
\thesisauthorA{Daniel Granerud}
\thesisauthorB{Jakob Sand Svarstad}
%\thesisauthorC{}
\thesissupervisor{Simon McCallum}
%\thesissupervisorA{} %second supervisor

\gmtkeywords{Bachelor, Thesis, Games, AR, cubes, IMT, cognitive, Augmented Reality}
\gmtdesc{Minigame environment with cubes in an augmented reality setting.}
\gmtnumber{18} % this is the number given to your project. May not be used  

\gmtoppdragsgiver{\GUC, Konstantinos Boletsis}
\gmtcontact{Konstantinos Boletsis, konstantinos.boletsis@hig.no, 12345678}




\thesisdate{\gucthesisdate}
\useyear{19.05.2014}

\gmtappnumber{} %number of appendixes
\gmtpagecount{} %currently auto calculated but might be wrong


\thesistitleNOR{cogARC}
\gmtkeywordsNOR{Bacheloroppgave, IMT, Spill, Utvidet Virkelighet, Kognisjon, Datasyn}
\gmtdescNOR{Denne oppgaven omhandler et milj\o{}  for \aa{} lage mini-spill som
kan brukes til kognitiv forskning. Teknologien som brukes er Augmented Reality.
Spillene l\o ses ved \aa{} flytte p\aa{} kuber med mark\o rer.}


% Create frontpages according to the school template
\makefrontpages

% Create project specific frontpage
\input{includes/cogarc_frontpage}

% Start counting pages after frontpage
\clearpage
\setcounter{page}{1}



% Preface
Thanks goes to Costas B. for being a great employer and giving us good feedback
throughout the project. He has been availible and able to give answers when we
needed it.

\tableofcontents
\listoffigures
\listoftables



% Start with content and count from 1.
\newpage
\setcounter{page}{1}
\pagenumbering{arabic}
% Put introduction here
\part{Introduction}
	
	\chapter{Introduction}
		\setcounter{page}{2}	% Needed to get numbering right
		\label{chap:introduction}
		cogARC is a bachelor project under Gj\o vik University College. We have been
working with \gls{Augmented Reality} to create a tool for cognitive research.
The tool has form as a game with logging functionality that can be used by
researchers. cogARC can typically be used by patients that has suffered stroke
or people with declining cognitive functionality. Doctors, medical staff and/or
researchers can use logging information to get an overview over the cognitiv
situation for the patient and can be enabled to track improvements or
deterioration.

\cite{GenVirtual}

\chapter{Development Team}
The development team consisted of Jakob Sand Svarstad, Per Kristian Warvik and Daniel Granerud.
All of them were taking the bachelor game programming course at Gj\o vik University College for the last three tears from the autumn of 2011. One of the members had never seen a line of code before attending Gj\o vik University College while the two others had some experience with coding.

Two of the team members have worked together continuously troughout the  troughout the their time at Gj\o vik University College, but as a team we have only worked together on one project. We found out that we worked well together on that project and decided that we would work well together on a bigger project as well.

\chapter{Externals}

\section{Employer}

\section{Supervisor}
Our supervisor for this project has been Simon McCallum.

% \chapter{Project Description}

% \section{Reasons for creating this thing}

% \section{Implementation plan}
% %How we planned to implement this

% \section{Group Organization}
% %How we as a group organized ourself and the work

% \section{Report Organization}
% %How the report is organized, what each chapter is and short about structural layout.

% \section{Project Goals}
% %Our goals in this project, how far we want the project to go and how complete we hope it will be. What parts will be implemented and such



% What is the setting of the problem? This is, in other words, the background. In some cases, this may be implicit, and in some cases, merged with the motivation below.
% What exactly is the problem you are trying to solve? This is the problem statement.
% Why is the problem important to solve? This is the motivation. In some cases, it may be implicit in the background, or the problem statement itself.
% Is the problem still unsolved? The constitutes the statement of past/related work crisply.
% Why is the problem difficult to solve? This is the statement of challenges. In some cases, it may be implicit in the problem statement. In others, you may have to say explicitly as to why the problem is worthy of a BTech/MTech/PhD, or a semester project, as the case may be.
% How have you solved the problem? Here you state the essence of your approach. This is of course expanded upon later, but it must be stated explicitly here.
% What are the conditions under which your solution is applicable? This is a statement of assumptions.
% What are the main results? You have to present the main summary of the results here.
% What is the summary of your contributions? This in some cases may be implicit in the rest of the introduction. Sometimes it helps to state contributions explicitly.
% How is the rest of the report organized? Here you include a paragraph on the flow of ideas in the rest of the report. For any report beyond 4-5 pages, this is a must.

	\chapter{Description and goals}
		\label{chap:description_goals}
		\subsection{What is cogArc?}
cogARC is a small framework that uses Unity and the Vuforia library to let you make a selection of mini games where the user can move a set of frame markers in real life and see them interact on a screen that augments the game.

\subsection{Background}
This project is part of the PhD project done by Konstantinos (Costas) Boletsis. Our aim is that this project will help him in documenting a users cognitive changes by playing an assortment of mini games. In these games the player moves digitally recognizable items around in the real world and an augmented reality device will show the player the goals for the game. We have created software that offer games for the player and logging information for our employer.

\subsection{Why this task?}
We chose this task primarily because it gave us an opportunity to work with several technologies that we previously had little or no experience with.
It also gave us good experience in the rapidly growing field of \gls{Augmented Reality}. 
One major factor that triggered us was that we were supposed to work with the 
\gls{Meta SpaceGlasses}. Unfortunately we never got them, but that was a
calculated risk. Instead we used an Android powered tablet.

\subsection{Target demographic}
The target demographic for the finished product is healthy people that are at high risk of cognitive decline. This can include, but is not limited to, seniors with mild cognitive impair and people that has suffered stroke.
Our employers hope is that by using this product he can measure the cognitive performance and possibly keep track of cognitive changes over time (if there is any).

\subsection{Goals}
The task we initially got was to create some mini-games using \gls{Augmented Reality}. Many of these games had many similar features, so we decided to have a different approach to the task. The goal we sat before us was to create a program, an environment, for making these games. We wanted the environment to enable our employer to manipulate the games we had created for him. It should also be able to use the features we had implemented to create new games. This should be done without changing a lot of source code and hopefully save valuable time.


\begin{description}
	\item[To be more precise, this is our goals:]\ 
	\begin{itemize}
		\item Functional \gls{AR} games.
		\item A manipulative environment.
		\item Implemented features available for creating similar games. 
		\item Interface through Unity's \gls{Inspector}.
	\end{itemize}
\end{description}


\subsection{Augmented Reality}
"Augmented reality (AR) is a live direct or indirect view of a physical, real-world environment whose elements are augmented (or supplemented) by computer-generated sensory input such as sound, video, graphics or GPS data. It is related to a more general concept called mediated reality, in which a view of reality is modified (possibly even diminished rather than augmented) by a computer. As a result, the technology functions by enhancing one’s current perception of reality. By contrast, virtual reality replaces the real world with a simulated one. Augmentation is conventionally in real-time and in semantic context with environmental elements, such as sports scores on TV during a match. With the help of advanced AR technology (e.g. adding computer vision and object recognition) the information about the surrounding real world of the user becomes interactive and digitally manipulable. Artificial information about the environment and its objects can be overlaid on the real world."\cite{WikiAugmentedReality}



\chapter{Organization}

\subsection{Implementation plan}
To implement this project we worked Monday to Thursday.
Friday was left to do other courses as each member of the class had signed up to at least one other class.
Exceptions to working on Monday to Thursday was made when it was needed.

On the scheduled days we sat together to coordinate what needed to be done and how progress was going so we could quickly re-plan when needed.
Most of the time we worked one by one although we sat together.
This let each member work at home if they wanted to go home to work or do the work at another time if they so wanted to.
If something needed multiple heads on the job this was then easily done as we was in the same room.

Delegation of work was done by the group as a whole where tasks was given to either the person most fitted to doing it based on previously shown skills, or if some one really wanted to do a task it would be given to that person. We did it like this as we saw this to be the fairest way.

At the end of each work week (Thursday) we held a short weekly meeting to asses what had been done by each member and what had been achived by the team.
We also wrote down what we planned to do the next week.
This way it was easy to track what had been done and what was to be done.
\todo{Is this enough?}

\subsection{Group Organization}
	\todo{Fill this one out better? Yes, a lot better would be nice.}
As a group we tried to organize ourselves so the workload would be as equal as possible. 
We had a group leader in name only as there was never a need to resolve a conflict within the group.
Although we tried to keep the workload equal it was split a bit between the three developers due to practicalities. 
For instance the work of doing the internal game logic was left to almost solely one developer as this required a lot of work with a few algorithms and little else.
We therefore saw it fitting to have one member of our group almost solely responsible to make it work whilst the two other developers did the other parts.
Although one developer was responsible for one thing we made sure that everyone on the team knew as much about every part as possible, in case one became sick or otherwise indisposed we would not be left with a big piece of work the others then would know nothing about.


\subsection{Report Organization}

\todo{How the report is divided and reasoning behind the dividing.}
\todo{How the report is organized, what each chapter is and short about structural layout.}

Our report is structured into 6 parts where each part contains chapters that are related to the encompassing part.

\begin{enumerate}
	\item Introduction.
	\item Design process.
	\item Development process.
	\item Product.
	\item Summary.
	\item Appendices.
\end{enumerate}

Introduction will introduce you to the project and its developers.


Design process will tell you how we designed and approached the project around our given requirements.


Development process will describe the theoretical part of the project as well show how we worked and what we did and with which tools.


Product will show you what we ended up with at the end of our development.


Summary will round up the project and thesis with a conclusion and afterword.


Appendices is at the end and will contain a bibliography, glossary, how we managed time and other neat tidbits.


\todo{Rewrite appendices at the end. Make this pretty.}

	\chapter{Objectives}
		\label{chap:objectives}
		\input{includes/cogARC_objectives}

	\chapter{Background and history}
		\label{chap:background}
		\input{includes/cogARC_background_and_history}

	\chapter{Related work}
		\label{chap:related_work}
		\gls{AR} is a technology that is starting to get popular. Because of this there
are not that many contributions to the field that is too similar to our project.
At the University of S\~{a}n Paulo, some academics have created a musical AR game
that has noticeable similarities with our work, they have called it GenVirtual
\cite{GenVirtual}. 

There exists AR games both for entertainment and for more purposeful
measures\cite{tan2010augmented}. In 2000, ARQuake stood out as the first fully
working outdoor AR game. It needed a lot of equipment attached to the body, so
it was never commercialized. Anyway, it generated a substantial interest in the
AR community. From 2005 and onwards more AR games have appeared. As
equipment have become cheaper, lighter and less space consuming, it has opened a whole new world of possibilities. 
Smartphones and tablets have brought forth
small devices with increasing processing power. With freely available libraries
as \gls{Vuforia}, many new applications are expected.



\part{Development}

	\chapter{Specification}
		\label{chap:specification}
		\section{Use Cases}

\paragraph{Use Cases}
\todo{Flow chart of how the user plays the game.}

\section{Conceptual class diagram}

\section{Functional requirements}
We did not get any functional requirements from our employer. There was
some requirements that would have to be there to make the program do what the
design document specified. The markers on the cubes would have to be seen. This
is definitely an issue when that functionality is created, but in our case we
have used the library \gls{Vuforia} for that. This means that we had no
influence on the direct observation on the cubes. Vuforia would analyze images
from the camera and give us positions of the different trackers when they were
observed. In our situation, the chance we had to influence here lies in how we
handle the input we get. We have written more about this opportunity in section
\ref{sect:input_handling}

\paragraph{Logging}
Our employer originally wanted us to log the hand movements of the user when the user is playing the game. With the META SpaceGlasses this is supposedly possible, but as of the writing of this report the SDK for the SpaceGlasses have yet to ship. In the beginning of the project we looked into doing this with the technology we were going to use for the project, Unity and Vuforia. We looked into this, but found that, neither Unity or Vuforia support any kind of tracking of hands. We brought this issue to our employer and after telling him of our findings we were told that we could then drop logging of the players hand movements altogether.
We tough this also meant that we would then drop all logging in the game (user high-score list not withstanding). But about two weeks before the due date it was brought to our attention that logging was essential for our employer to have. So with very little time left of the project we suddenly had to add logging of the user interaction and store the information in a way that was both accessible and useful to our employer.
To fullfill these requirements we had the help of our supervisor Simon McCallum to set up a web server where we could post our data to.
The data we send is username (chosen by the user in the game, we use a special identifier if the user-name is not set), what score the player currently has, what game is being played, what level of that mini-game is being played, event type, event data and lastly time. 

\todo{Make Jakob expand this segment.}

\section{Operational requirements}
Here is op

	\chapter{Tools}
		\label{chap:tools}

		\section{Unity}
			\todo{Why...?}
\todo{Generally how it works. Generally how it does not work. Generally how it make people cry.}
Unity Unity Unity, what is it and why do we hate it.

Unity is a game engine and development enviroment.
It features a intergrated work enviroment with good tools that promotes rapid workflows to create the games you wish to create without worrying about the underlying structure.
With its multiplatform deployment tools and intergrated asset store it has become very popular for games.
Supporting both 2D, 3D and animating makes so that you dont need other programs to create what you wish to create.

Getting into Unity for the first time is greatly aided by the many video tutorials that Unity has created, showing how to use the interface, basics of scripting and programming, how to add sound and much more.



		\section{Vuforia}
			\todo{Why...?}
\todo{Generally how it works. Generally how it does not work. Generally how it make people cry.}
Unity Unity Unity, what is it and why do we hate it.

Unity is a game engine and development enviroment.
It features a intergrated work enviroment with good tools that promotes rapid workflows to create the games you wish to create without worrying about the underlying structure.
With its multiplatform deployment tools and intergrated asset store it has become very popular for games.
Supporting both 2D, 3D and animating makes so that you dont need other programs to create what you wish to create.

Getting into Unity for the first time is greatly aided by the many video tutorials that Unity has created, showing how to use the interface, basics of scripting and programming, how to add sound and much more.



		\section{MonoDevelop}
			MonoDevelop is the integrated development environment (IDE) that comes with Unity. It supports writing in C\#, Unity JavaScript and Boo.
MonoDevelop is developed by Xamarin \cite{xamarinRef} as an open source integrated development environment for Windows, Linux and OS X.
Its primary focus is on C\# and other .NET languages.


MonoDevelop offers debugging features such as breakpoint, stepping trough code one line at a time, tracking of variables, call stack of functions and more.
It also have features to help with programming such as code completion and solution overview. Alongside all this it also supports add-in's for adding support for other languages.


\begin{wrapfigure}{l}{0.3\textwidth}
	\capstart
	\centering
	\vspace{-10pt}
	\includegraphics[width=0.28\textwidth]{images/MonoDevelopLogo.png}
	\vspace{-20pt}
	\caption[MonoDevelop IDE Logo]{{M}ono{D}evelop {IDE}}
	\label{fig:monodevelop}
	\vspace{-10px}
\end{wrapfigure}

For us MonoDevelop gave mixed results. 
I have no doubt that it is a good IDE for working with C\# or a .NET language, but the UnityScript integration is poorly done.
The auto complete functions was flimsy at the best of times, most of the time it never activated and when it did it did not give the correct results. 
There has also been times when trying to get a public variable have resulted in a error because the variable has yet to be defined while it is available two lines above with no change in the scope.
We believe these problems are more due to Unitys implementation of JavaScript and not a error on MonoDevelops part, so if one are to use MonoDevelop we would recomend to shy away from Unitys JavaScript and use C\# instead.
Ending with on a positive note it had the best "Watch" feature we have come accros, it made tracking of several variables at a time really simple even if that variable was part of another script or class.	

\todo{Does this need more?}

		\section{GIT}
			We have been using GIT for version control in this project. This is a version
control system we had some prior experience with, so we wanted to try it for 
this project as well. 

GIT works best with plain text files. With Unity there is quite many binary
files as well. At first we were not entirely sure how much of the files that
were binary and if GIT could work for us. After some googling we found that 
this was possible. On Unity's webpage they instruct on how to use the version
control system called Subversion \cite{SubversionControl}. Together with some
help from Stackoverflow.com\cite{gitVersionControl} we figured out how to set 
up the project with GIT.

Unity produces .meta-files for each file in the project. We do not know 
entirely how these works, so we met some problems that we suspected had its
origin in these files. Because of this we have tried both to have them in
the GIT repository and not. The solution to that specific problem was not
clear when we got it working, so we do not know whether the metafiles were
problematic or not. The post at Stackoverflow explained that the .meta-files 
could be set to hidden. This implies that the files should not be part
of the GIT repository. Anyway, now they are because it seemed to help with the
problem we had, and it has not given any other problems later.


\begin{wrapfigure}{l}{0.3\textwidth}
	\capstart
	\centering
	\vspace{-10pt}
	\includegraphics[width=0.28\textwidth]{images/git}
	\vspace{-5pt}
	\caption[GIT logo]{GIT}
	\label{fig:git}
	\vspace{-10pt}
\end{wrapfigure}

Many of the pure binary files has not been necessary to upload. Those that
has, like images, has often been of a type that didn't need to be updated
often. This means that most of the content in the repository could be plain
text and therefore we have been able to use GIT without further hassle.

\begin{wrapfigure}{r}{0.3\textwidth}
	\capstart
	\centering
	\vspace{-10pt}
	\includegraphics[width=0.28\textwidth]{images/github}
	\vspace{-5pt}
	\caption[{G}it{H}ub logo]{{G}it{H}ub}
	\label{fig:github}
	\vspace{-10pt}
\end{wrapfigure}

We have used GitHub \cite{GitHub} as our online repository server. By using
this service we could easily share the work we have done. That we could use
an external server like this has been important for us when choosing version
control tool. Especially in one occasion this proved to be very smart. One of
us encountered a bug in Unity that were quite severe. The Unity3D program
crashed, and by some unknown reason, it deleted a lot of files. The files that
were removed were all of the project files + some other files in the parent
directory. Many totally unrelated files were deleted and 4 days work that was
not committed to the server were lost. But fortunately enough, he could
download the latest version from GitHub and use the last days lessons to write
the lost changes a whole lot faster.

When we have been using GIT we have used the commandline tool and also the GUI
program for GIT called Sourcetree \cite{SourceTree}.

The workflow we have been using has been very simplistic. We are aware that
there are good workflows that we could have used, but we have chosen to only
have a master branch that everyone pushes to. During the project we were told
about a way that worked for another group were they used pull requests and in
this way added an extra layer of security of code integrity. We tried this out
for a short time, but since none of us had experience with this way of working,
we discarded it after a relatively short time. We were working quite close with
many small and rapid changes that we shared with everyone. If another group
member should accept this every time, we think that a lot of unnecessary time
could have been wasted on this. The model is not bad, but we have come to the
conclusion that it didn't fit our way of working very well. If we had started 
to do it from the beginning, we could maybe have done it, but the effort may
still have not been worth it. Anyhow, we are glad to learn about other ways of
working that can be useful later.

	\chapter{Schedule}
		We planned the development project and ended up with the gantt diagram shown in Figure \ref{fig:preplan_gantt}:

\begin{figure}[h]
	\capstart
	\centering
	\includegraphics[height=\textwidth, angle=270]{preplan_gantt_diagram}
	\caption[Preplan Gantt diagram]{Our initial gantt diagram for the project.}	\label{fig:preplan_gantt}
\end{figure}


Afterwards we see that the process has been like this:

\todo{Insert gantt diagram of our dev-process here.}

\section{Planning vs reality}%Or how we worked compared to the plan
Our development cycle deviated a lot from the original plan due to a multitude of factors.
For instance we could have allocated less time to the "Create application enciromnet and menu UI" and "Recognice cubes" which each were allocated 7 days in the plan, but had we known more about Unity or the Vuforia plug-in in Unity then we would have known that Unity have methods and functions for creating in-game UI and to make the Vuforia plug-in working were just adding a \gls{prefab} for the camera.
We spent the time allocated to those time slots to creating the framework instead and this additional time shifted a lot of the sub-sequential work a week ahead of time.
Task number 10 (Real world orientation (for besides and upon)) was moved to the beginning of the project by developer Jakob and was done as part of him learning Unity.

The most drastic change in the time schedule is in item 11-15.
After learning some of the inner workings of Unity and how to properly set up a game in Unity it became rapidly apparent that we had taken a somewhat naive approach on it. This was due to our lack of knowledge of how much a game engine can do for us.
Using Unity we had a almost complete mini-game enviroment from the beginning. As such we only had to program what interaction we wanted to happen when cubes collided.
Our time and efforts were then spent on making the interaction between cubes and the player to what we wanted instead of having to focus on making the basics of a game world working.


	\chapter{Development process}
		%Write aboot the development process
%Fuck yes


	\chapter{UnityScript}
		\label{chap:UnityScript}
		In Unity a user can use several different languages to script their games. Boo, C\#, JavaScript and Shader (used for writing shaders).
While Boo and C\#  use the standard implementation and is easily referenced the JavaScript used in Unity is not.\cite{WikiScriptVSScript}

Unitys JavaScript is quite different in many ways from the more standardly used JavaScript that is based on ECMAScript\cite{ECMAscipt} which is a standardized scripting
language standardized by Ecma International.

Some of the major and notable differences between Unitys JavaScript (henceforth named UnityScript) and ECMASCripts JavaScript (henceforth referred to as JavaScript):

\begin{itemize}
	\item Classes.
	\item Different privacy.
	\item Not JavaScript library compatible.
\end{itemize}

\subsection {Classes}
JavaScript have no concept of classes as it is a prototypical language, meaning it will have the properties that is assigned to it, but is not bound to them.
Properties can be added or deleted at will in runtime, this is true for both members of the object and functions. 
UnityScript is much more object oriented with strongly typed and defined classes that cannot be changed in runtime (the exception to this rule is reflection).
This makes it difficult to work with UnityScript if you have little experience using it, or have yet to find out that the difference between JavaScriptand UnityScript.
UnityScript have also removed the prototypical properties of JavaScript by implementing it as a class based language. 
Meaning objects can not get additional functionality or be completely redefined at runtime. 
In UnityScript each .js file is implemented as a class by default. 
So when a developer creates a new .js file the Unity compiler and serialization system will add a class definition around the data within the file by default, so to call a function within a .js file you write "filename".function.
This however is not documented and gave us problems in the starting phase when we were learning UnityScript as there was no obvious way to get data or functions from other files.

\subsection {Different privacy}
UnityScript, being structured like other object oriented languages have the usual levels of privacy of members within a class being marked as either Private, Protected or Public. 
JavaScript have per function privacy. 
Meaning a variable declared in a function is available in the entire function but not outside of the function.
But sometimes the compiler gets this confused or ModoDevelop get this confused as we have had problems with a variable being accessible in one part of the function but not in other parts.
This has been the case with member variables of the script and variables in the function not being available, this led to having some times having to create temporary variables to store the original variable to be able to use it in the fucntion.
%
%\subsection {var as a keyword is required}
%If you do not use the keyword var in JavaScript the variable becomes a global variable as the var keyword is used to denote scoping. In UnityScript the var keyword is required when creating a variable to make sure the scoping is always the intended scope.

\subsection {Not JavaScript library compatible}
UnityScript being so different from JavaScript makes other premade JavaScript library non-compatible with UnityScript, which also makes it really difficult to find reference help. 
One half of the built in library in Unitys Monodevelop is Unitys own library and the other half is made using the .NET framework from MicroSoft.
This was a small problem when we looked into using {JSON} objects in our program and wanted to use a library.
Since {JSON} is a part of the JavaScript language and specification we thought that it was also implemented in UnityScript.
This was not the case, but we are not the first to notice this and there are now several {JSON}-parser libraries available created for Unity in both C\# and in UnityScript.
One such library is Boomlagoon\cite{Boomlagoon.JSON} that we ended up using for our project.


	\chapter{Unity Serialization}
		\label{chap:UnitySerialization}
		Unity operates with two memory spaces. A native memory space belonging in to the C++ side of the code and the more managed DLL side that comes from scripts.
The managed side memory is where all the data from scripts that users make are put.
To use the data Unity does a serialization and deserialization during a assembly reload. What happens is that it pulls all the data out of the managed side, creates an internal representation of the data on the C++ side, destroys all the memory from the managed side, reloads the assemblies and then
re-serialize the data from C++ into the managed side.
The serialization happens when the system updates its assembly files which is when the Editor or the system reloads an updated assembly, when the user enter or exits play mode and when the scene is loaded or saved.
A serialization can therefore happen rather frequently or rarely, depending on the work-flow of the user and what the user is currently doing.

Unity is only able to serialize basic data types plus those already defined within Unity such as the GameObject class, Transform object and such.
What this means is that when Unity does not know explicitly that the data are to be serialized or if it is unable to serialize it, it will simply be destroyed. Most classes will not be serialized unless they have data that Unity can recognize is being used, struct can not be serialized, private members to a script will not be serialized unless Unity finds a explicit reference to it outside of the script, arrays containing complex data is likely not to get serialized.
However, there are a few ways to make sure Unity knows that the data shall be preserved not destroyed:

\begin{enumerate}
	\item Make the data field public
	\item Mark the field as serialize-able (@SerializeField in Unity JavaScript and [SerlializeField] in C\#)
	\item Mark the class as serialize-able (@Serialize in Unity JavaScript and [Serialize] in C\#)
	\item Make a class that derives its base type from ScriptableObject
\end {enumerate}

By having the data marked in either of those ways or a combination of them will ensure that Unity will try to serialize it if it can.
But there are a few things that Unity can not easily serialize, even if it adhere to the previously mentioned guidelines, and most notably it can not serialize an array of normal objects where the objects contain data that Unity does serialize, not even a partial serialization where only some data is destroyed.
It can serialize an array of objects that is derived from ScriptableObject. What the serialize will do is serialize each object in the
array individually and put a pointer into the array. To make this work the ScriptableObject derived class has to be in its own
C\# file.
Unfortunately for us we are using Unity JavaScript in this project, and getting C\# and JavaScript to work nicely
together is not a trivial task in Unity.

The Unity serialization gave us as a group a lot of troubles, specifically with the data we wanted saved for how each cube is designed for each separate game.
For the cubes design we made a class named BoxDesign that was intended to contain all the data and functionality needed to control and set the design for the cubes in the game.
By following the guidelines from Unity on how to make the class serialize we made the class derive from ScriptableObject and marked all non-public fields with @SerializeField and marked the entire class with @Serialize, but no matter what we tried to do the array we put the data into did not survive the assembly reload.
It took most of the group the better part of three weeks of work and research on the matter to find a solution.
 After an enormously amount of attempts and fixes we ended up with a solution that while maybe not the best is reliable and functional. The solution was to encode the design into a JSON string and store the resulting string since strings are supported by the Unity serialization, then when we want to get the data for the cubes we decode the JSON string back into our BoxDesign class and store them in a array that we can then use in the rest of our code.
 

	\chapter{Development changes and decisions}
		\label{chap:Developmentchangesanddecisions}
		Following is a list of decisions and changes we as a group made during the development of this project in chronological order.

\begin{enumerate}
	\item 09.jan: As many as possible mini games should be created in a dynamic way.
	\item 09.jan: Graphics for the system does not have to be more advanced than what is shown in the design document.
	\item 09.jan: The cubes need only to have basic collision between others to achieve the functionality we need for the games.
	\item 09.jan: Our focus should be on making as many games as possible within the framework and not on making  a few perfect ones.
	\item 24.jan: We decided to shift our focus to be more on the framework itself and not the games within the framework.
	\item 30.jan: Costas wants us to implement a variation of the mobile app Wooords to replace a few of the other mini games.
	\item 31.jan: We decide to look into the implementation of Wooords at a later date in the development as it is a interesting game and will enhance the finished app since it is not similar to the other mini games.
	\item 10.feb: The group, along with Costas have decided to drop the memory mini game and the path mini game described in the design document.
	\item 17.feb: In the design document it implies that the loading screens should have all time leader board and a high score for the player. This is not consistent with the rest of the design document nor with our planning document and it was decided to instead have a high score for the current player on the loading screen instead.
	\item 06.mar: The design of the boxes is set in the inspector during the making of the game in Unity, the design can therefore not be changed during runtime.
	\item 10.mar: We decided to look into detached / integrated use of a AR library.
	\item 18.mar: Pictures that will be used as textures on the cubes must be in a folder called "/Resources/BoxDesign/" to be able to properly load them since we are not able to extract the path of where a texture is stored on disc.
	\item 27.mar: The levels for mini games will be procedural generated unless you want to make a Wooords mini game, where the levels will be read from file since these levels can not be easily randomly generated.
	\item 29.apr: The restart button have been removed from the pause screen.
\end{enumerate}

\section{Framework instead of games}
In the beginning we wanted to make sure that we would be able to create all the games described in the design document, and after looking into how we should structure the programs and what scripts and the like could be shared between each of the mini games we realized that on the programming side of them there was little difference. We therefore decided upon focusing making a framework to work within instead of making each mini game special.

\section{Implementing a word game instead}
Shortly after beginning development and starting testing of the \gls{Frame Marker} we realized that the mini game \#5: Memory cubes as it was described in the design document was not implementable. Playing the game would make the player turn the cube upside down in the best case scenario or covering the frame marker and then turn the cube in the worst case. This caused us a lot of problems because of the way that Vuforia controls its frame markers. What Vuforia does when it is not tracking a frame marker or it loses the tracking is to deactivate the frame marker in the scene hierarchy which makes us unable to access any information contained within both the frame marker object itself and its children. This meant that we had no reliable way to detect if the player turned a cube to see the hidden mark under the cube or if it was simply re-detected by the tracker due either by the player obscuring the marker or something else obscuring the marker or the marker just being lost for a few frames. We brought this issue to our employer and after a short discussion we came to the conclusion to implement our own version of the mobile game Wooords and not make the mini game \#5: Memory cubes and also to not make mini game \#6: Path as the game was deemed % nu nu nu
Woords is a game where the player is presented with a jumble of words and a theme. The goal is to find as many or all the words in the level by combining the letters to make words.


\section{Dropping leader-board}
In the original design document we got there was mention of having the loading screen both instructions for the game and a global leader-board. Unity does not work well with net based activities 

\section{Putting textures in a special folder}
This decision is due to a limitation in Unity. When you place a object in the object field in the inspector it will give us a new instance of that object, in this case a texture. What it does not give us is where it is stored. This would not be a problem if Unity was able to serialize our data, but since we are storing our objects as a JSON object due to Unity not being able to serialize it we have to store it ourself and the name of the texture is the only data we can rely on. The name however does not include the path of where the object is stored, so when we are extracting the data back from JSON objects into textures we only have the name of the texture. So to avoid any confusion of duplicate names and avoiding searching for the texture we constrict the location of where it can be placed into a sub-folder in the Resources folder of Unity.

\section{Procedural generation over predetermined levels}
In the beginning of the development we were uncertain about whether we should create levels for the mini games with code or if it should be crafted by hand. We initially went for the possibility of both, however after discussing it with our employer we came to the conclusion that as many of the games as possible should have generated levels. The only mini game that is not suitable for random generation is mini game \#7: Wooord game, the Wooord game has a single file for each set up starting with the overall theme (example is: animal, this means that all the words are animal related), followed by a comma separated sequence of what words should be on the cubes and on the line under is a comma separated list of words that can be created with those words. To create more levels the file just have to continue with a new sequence of letters for the cubes to have and a new line of words that can be created.

\section{The restart button has been removed from the pause screen}
\todo{Jakob kan skrive om hvorfor den f\o{}rst \o dela alt, men at vi senere fikset den.}



\part{Summary}

	\chapter{Afterword}
		\label{chap:afterword}
		\input{includes/cogARC_afterword}


\todo{Remove this when people know how to use glossaries or before delivery}
Here is an example on how to use glossaries. We can for example use the word
\gls{ECMAScript}. Now we get the text ECMAScript with a link to the glossary
list. In the glossary list there will also be a link to where this is 
referenced in the document. The same applies for \gls{serialization}.

Example of references inside the document:
	Here is a reference to chapter \ref{chap:related_work}.




% References
\bibliographystyle{classes/gucthesis}
\bibliography{includes/cogARC_bibliography}

% Glossary list
\printnoidxglossary[sort=word]


\part{Appendices}

\appendix %after this line all chapters will have leters instead of numbers
%\appendixpage
%\addappheadtotoc

% Examples
% \input{includes/example_packages}

% \input{includes/example_structure}

% \input{includes/example_meetings}

% \GUC, 
% \comment{so what}

% \todo{task \#1}
% This is a todo-task we need to get done.


\end{document}