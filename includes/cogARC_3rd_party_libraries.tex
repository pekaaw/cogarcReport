\paragraph{PlayerPrefsX}
PlayerPrefsX (also known as ArrayPrefs 2)\cite{PlayerPrefsX} is a library created by the Unity community to enhance the built in Player Preference class of Unity.

PlayerPrefsX expands Unitys' way of storing player preferences or other data that the programmer want to persist between sessions.
Unity without PlayerPrefsX supports saving and getting float values, int values and strings. Altough this is all that is needed in most cases, it is sometimes practical to be able to store a bit more complex data.

PlayerPrefsX is able to store vectors, quaternions, arrays and more. 
We used this library to store the list of high scores for each game. 
By using PlayerPrefsX IntArray storing it was very easy to store a array of ten high scores for each game instead of having to either make a complex string to store it or fining another way of storing the scores.

\paragraph{Boomlagoon.{JSON}}

\begin{wrapfigure}{r}{0.3\textwidth}
	\capstart
	\centering
	\vspace{-20pt}
	\includegraphics[width=0.28\textwidth]{images/Boomlagoon.png}
	\vspace{-20pt}
	\caption[Boomlagoon {L}td.]{Boomlagoon {L}td.}
	\label{fig:boomlagoon}
	\vspace{-10pt}
\end{wrapfigure}

Due to some problems we encountered with the \nameref{sec:UnitySerialization},
we needed to be able to save and restore a structure in another way than the
default one. We spent weeks trying to solve this problem. When we stopped
looking for better ways to solve it, we had ended up on converting the 
structure to {JSON} and save the whole thing as a string.
Unity has some support for {JSON} with it's library called Simple{JSON} 
\cite{SimpleJSON}. Simple{JSON} have the drawback that it only stores data
as a string, which will make the restoration difficult. Instead we found
another implementation of JSON that manage to handle different types.

Boomlagoon.{JSON} have two main types that we have used. That is {JSONO}bject
and {JSONV}alue. A typical {JSONO}bject holds other {JSONO}bjects and/or
{JSONV}alues. A {JSONV}alue holds information about type and content.

By using Boomlagoon's implementation of {JSON} we were able to store the
structure as a {JSONO}bject, which again can be stored and sent as a {S}tring.
This solved our problem since {U}nity are able to serialize and deserialize 
{S}trings. We think the solution is somewhat of a hack and we wish we had found
a better way to solve the problem. Unfortunately the search for a better 
solution was taking so much time that we just had to go ahead and use the best
solution we had found.

In order to use the library, we had to be sure that it was compiled before it
was ever used. We got some problems due to this fact and after some research,
we found that we could precompile it ourselves. Boomlagoon.{JSON} is a stand-
alone C\# implementation. Because of this we were able to import it into a
Microsoft Visual Studio Project and compile it as a DLL. None of us had done 
this before so it was some useful learning in it.

We tried the same thing with a Vuforia script that also had to be compiled 
before it was used. This was not working since that script was not stand-alone.
The solution there was to move the script to another folder.
