MonoDevelop is the integrated development environment (IDE) that comes with Unity. It supports writing in C\#, Unity JavaScript and Boo.
MonoDevelop is developed by Xamarin \cite{xamarinRef} as an open source integrated development environment for Windows, Linux and OS X.
Its primary focus is on C\# and other .NET languages.


MonoDevelop offers debugging features such as breakpoint, stepping trough code one line at a time, tracking of variables, call stack of functions and more.
It also have features to help with programming such as code completion and solution overview. Alongside all this it also supports add-in's for adding support for other languages.


For us MonoDevelop gave mixed results. 
I have no doubt that it is a good IDE for working with C\# or a .NET language, but the UnityScript integration is poorly done.
The auto complete functions was flimsy at the best of times, most of the time it never activated and when it did it did not give the correct results. 
There has also been times when trying to get a public variable have resulted in a error because the variable has yet to be defined while it is available two lines above with no change in the scope.
We believe these problems are more due to Unitys implementation of JavaScript and not a error on MonoDevelops part, so if one are to use MonoDevelop we would recomend to shy away from Unitys JavaScript and use C\# instead.
Ending with on a positive note it had the best "Watch" feature we have come accros, it made tracking of several variables at a time really simple even if that variable was part of another script or class.	

\todo{more...}
