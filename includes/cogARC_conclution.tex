\subsection{What we set out to do}

\begin{description}
	\item \todo{Comment on them. They are only copied from the planning document...}

	\item[Result goals]\ 
	\begin{itemize}
		\item 7 games.
		\item Adaptable framework.
		\item Viable bachelor thesis.
		\item Combine physical cubes and digital structures.
	\end{itemize}
	From the initial plans some game ideas were dropped due to technical issues and some were dropped because they were found to be unnecessary or uninteresting and some were added to take their place. 
	This selection-process was done over time early in the project and we completed all the games we decided to do. 
	We also made code for one of the games we dropped, but this game was not put into the application and was never properly tested. 
	We can create a new simple game type in 1-2 days.\\
	The framework we made is flexible beyond what was required to make the requested games. 
	The rules of games can easily be changed in Unity. 
	There are several options for changing the difficulty of games, such as setting time limits, setting number of cubes used to form solutions and setting number of levels. 
	For the \nameref{game:wo0ord_game} the file containing the tasks can easily be changed. 
	The game description can be changed and the score system can to some degree be tweaked in the inspector. 
	The main-menu is not as flexible and it is the most problematic area when wanting to expand the application to have more than the 6 base games or more series than the all game series. 
	In the scripts it's fairly easy to create long series of games. It takes a little more work to create buttons in the main-menu for starting them.\\
	\todo{did we do enough work?}
	yes we believe we did.\\

	We turned the volatile environment seen by the cameras of the real world
	into fairly stable arrays and lists and tested them against both preset 
	and randomly created states. Some buggy behavior occurs, but most of this
	is because the images from the camera are too bad for the trackers or because
	of Vuforias tracking algorithms.\\

	\item[Effect goals]\ 
	\begin{itemize}
		\item Support research.
		\item Experience with a game engine.
		\item Experience with a longer development process.
		\item Experience with digital space linked to reality.
	\end{itemize}
	We cannot really say at this point how well our product will support the research, but our employer stated that he was pleased with the product and we were able to meet most of the requirements for logging events. 
	So we reckon it will be at least a little useful.\\
	We haven't had a lot experience with large projects and they have all been in different environments, coding languages and with different tools. 
	So we had a lot of trouble in the beginning of the project figuring out how to structure, set up the scene hierarchy, where to put scripts and how to make scripts communicate.
	As we implemented new features we went over the structure several times, improving it a little each time. 
	The Unity project still shows signs of structural weaknesses we chose not to remove because it would mean a lot of refactoring-work. 
	We believe that in future projects in Unity and similar we will be able to set much better structure because of the experience we gained in this project.\\
	We didn't really explore the AR field as much as our supervisor wanted us to, but we studied the possibilities. 
	If the technology had been better we might have gone much deeper in this direction.


\end{description}


\subsection{What we did}

\subsection{How we did it}
\todo{Evaluate the groupwork.}
As a group we worked very well together.
We worked together very closely throughout the entire development by sitting in a storage room in Mustad we all became good friends and had very good synergy between us.
Disagreements were few and far between and none of them escalated to be more than heated discussions due to minor miss-communications between two group members.

\subsection{Reflection}
	\todo{End status report: What works, what was dropped, what could need more time.}
In the end we got everything we wanted working.
We implemented all games mentioned in the design document as good as we possibly could except for the two games that we decided along with our employer to drop.
One of them was dropped due to limitations with the technology we are using and the other one was dropped as our employer found it to be unnecessary if we implemented the \nameref{game:wo0ord_game}.

Although we tested everything during development we would still like to have more time to test on more devices and have more playtesters.
We would also like to completely redo everything we coded.
Some parts of the code stills shows that we had little experience with Unity and our code hierarchy definitely could do with a overhaul.
But even with this we have a product that can be easily added on to create code to support more games later or to change the games already coded.

In the end we have a project that we are pleased with, where we learned a lot about a new progams, a new coding language and how it is to work on a big project.
