\section{What is cogArc?}
cogARC is a small framework that uses Unity and the Vuforia library to let you make a selection of mini games where the user can move a set of frame markers in real life and see them interact on a screen that augments the game.

\section{Background}
This project is part of the PhD project Konstantinos (Costas) Boletsis. Our aim is that this project will help him in documenting a users cognitive changes by playing a assortment of mini games where the player moves items with markers around in the real world and an augmented reality enabled device will show the player the goals for the game. Our contribution will be to create some of the software for doing this.

\section{Goals}
The task we initially got was to create some minigames using \gls{Augmented Reality}. Many of these games had many similar features, so we decided to have a different approach to the task. The goal we sat before us was to create a program, an environment, for making these games. We wanted the environment to enable our employer to manipulate the games we had created for him. It should also be able to use the features we had implemented to create new games. This should be done without changing a lot of source code and hopefully save valuable time.


\begin{description}
	\item \todo{Maybe we should have the goal-list another place, or comment on them instead? They are only copied from the planning document}

	\item[Result goals]\ 
	\begin{itemize}
		\item 7 games.
		\item Adaptable framework.
		\item Viable bachelor thesis.
		\item Combine physical cubes and digital structures.
	\end{itemize}
	\item[Effect goals]\ 
	\begin{itemize}
		\item Supporting research.
		\item Experience with a game engine.
		\item Experience with a longer development process.
		\item Experience with digital space linked to reality
	\end{itemize}
\end{description}


\section{Augmented Reality}
"Augmented reality (AR) is a live direct or indirect view of a physical, real-world environment whose elements are augmented (or supplemented) by computer-generated sensory input such as sound, video, graphics or GPS data. It is related to a more general concept called mediated reality, in which a view of reality is modified (possibly even diminished rather than augmented) by a computer. As a result, the technology functions by enhancing one’s current perception of reality. By contrast, virtual reality replaces the real world with a simulated one. Augmentation is conventionally in real-time and in semantic context with environmental elements, such as sports scores on TV during a match. With the help of advanced AR technology (e.g. adding computer vision and object recognition) the information about the surrounding real world of the user becomes interactive and digitally manipulable. Artificial information about the environment and its objects can be overlaid on the real world."\cite{WikiAugmentedReality}
