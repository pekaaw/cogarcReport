\section{What is cogArc?}
cogARC is a small framework that uses Unity and the Vuforia library to let you make a selection of mini games where the user can move a set of frame markers in real life and see them interact on a screen that augments the game.

\section{Background}
This project is part of the PhD project done by Konstantinos (Costas) Boletsis. Our aim is that this project will help him in documenting a users cognitive changes by playing an assortment of mini games. In these games the player moves digitally recognizable items around in the real world and an augmented reality device will show the player the goals for the game. We have created software that offer games for the player and logging information for our employer.

\section{Why this task?}
We chose this task primarily because it gave us an opportunity to work with several technologies that we previously had little or no experience with.
It also gave us good experience in the rapidly growing field of \gls{Augmented Reality}.

\section{Target demographic}
The target demographic for the finished product is healthy people that are at high risk of cognitive decline. This can include, but is not limited to, seniors with mild cognitive impair and people that has suffered stroke.
Our employers hope is that by using this product he can measure the cognitive performance and possibly keep track of cognitive changes over time (if there is any).

\section{Goals}
The task we initially got was to create some mini-games using \gls{Augmented Reality}. Many of these games had many similar features, so we decided to have a different approach to the task. The goal we sat before us was to create a program, an environment, for making these games. We wanted the environment to enable our employer to manipulate the games we had created for him. It should also be able to use the features we had implemented to create new games. This should be done without changing a lot of source code and hopefully save valuable time.


\begin{description}
	\item[To be more precise, this is our goals:]\ 
	\begin{itemize}
		\item Functional \gls{AR} games.
		\item A manipulative environment.
		\item Implemented features availible for creating similar games. 
		\item Interface through Unity's \gls{Inspector}.
	\end{itemize}
\end{description}


\section{Augmented Reality}
"Augmented reality (AR) is a live direct or indirect view of a physical, real-world environment whose elements are augmented (or supplemented) by computer-generated sensory input such as sound, video, graphics or GPS data. It is related to a more general concept called mediated reality, in which a view of reality is modified (possibly even diminished rather than augmented) by a computer. As a result, the technology functions by enhancing one’s current perception of reality. By contrast, virtual reality replaces the real world with a simulated one. Augmentation is conventionally in real-time and in semantic context with environmental elements, such as sports scores on TV during a match. With the help of advanced AR technology (e.g. adding computer vision and object recognition) the information about the surrounding real world of the user becomes interactive and digitally manipulable. Artificial information about the environment and its objects can be overlaid on the real world."\cite{WikiAugmentedReality}



\chapter{Organization}

\section{Implementation plan}
	\todo{How we planned to implement this}

\section{Group Organization}
	\todo{Fill this one out better? Yes, a lot better would be nice.}
As a group we tried to organize ourselves so the workload would be as equal as possible. We had a group leader in name only as there was never a need to resolve a conflict within the group.
Although we tried to keep the workload equal it was split a bit between the three developers due to practicalities. 
For instance the work of doing the internal game logic was left to almost solely one developer as this required a lot of work with a few algorithms and little else.
We therefore saw it fitting to have one member of our group almost solely responsible to make it work whilst the two other developers did the other parts.
Although one developer was responsible for one thing we made sure that everyone on the team knew as much about every part as possible, in case one became sick or otherwise indisposed we would not be left with a big piece of work the others then would know nothing about.


\section{Report Organization}

\todo{How the report is divided and reasoning behind the dividing.}
\todo{How the report is organized, what each chapter is and short about structural layout.}

Our report is structured into 6 parts where each part contains chapters that are related to the encompassing part.

\begin{enumerate}
	\item Introduction.
	\item Design process.
	\item Development process.
	\item Product.
	\item Summary.
	\item Appendices.
\end{enumerate}

Introduction will introduce you to the project and its developers.


Design process will tell you how we designed and approached the project around our given requirements.


Development process will describe the theoretical part of the project as well show how we worked and what we did and with wich tools.


Product will show you what we ended up with at the end of our development.


Summary will round up the project and thesis with a conclusion and afterword.


Appendices is at the end and will contain a bibliography, glossary, how we managed time and other neat tidbits.


\todo{Rewrite appendices at the end. Make this pretty.}
