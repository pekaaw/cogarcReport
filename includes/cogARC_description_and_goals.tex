\subsection{What is cogArc?}
\label{subsec:what_is_cogarc}
cogARC is a small framework that uses Unity and the Vuforia library to let you make a selection of mini games where the user can move a set of frame markers in real life and see them interact on a screen that augments the game.

\subsection{Background}
\label{subsec:background}
This project is part of the PhD project done by Costas Boletsis. Our aim is that this project will help him in documenting a users cognitive changes by playing an assortment of mini games. In these games the player moves digitally recognizable items around in the real world and an augmented reality device will show the player digital goals and results in the game. 

\subsection{Why this task?}
\label{subsec:why_this_task}
We chose this task primarily because it gave us an opportunity to work with several technologies that we previously had little or no experience with.
It also gave us good experience in the rapidly growing field of \gls{Augmented Reality}. 
One major factor that triggered us was that we were supposed to work with the 
\gls{Meta SpaceGlasses}. Unfortunately we never got them. They were expected to be released in December
 2013, but unfortunately the release of have been pushed forward to July 2014.
 However this was a calculated risk. Instead we have been using an Android powered tablet as well as our own smartphones.

\subsection{Target demographic}
\label{subsec:target_demographic}
The target demographic for the finished product is healthy people that are at high risk of cognitive decline. This can include, but is not limited to, seniors with mild cognitive impair and people that has suffered stroke.
Our employers hope is that by using this product he can measure the cognitive performance and possibly keep track of cognitive changes over time (if there is any).

\subsection{Goals}
\label{subsec:goals}
The task we initially got was to create some mini-games using \gls{Augmented Reality}. Many of these games had many similar features, so we decided to have a different approach to the task. The goal we sat before us was to create a program, an environment, for making these games. We wanted the environment to enable our employer to manipulate the games we had created for him. It should also be able to use the features we had implemented to create new games. This should be done without changing a lot of source code and hopefully save valuable time.


\begin{description}
	\item[To be more precise, our goals are:]\ 
	\begin{itemize}
		\item A set of functional \gls{AR} games.
		\item A manipulative environment.
		\item Implementing features for creating similar games. 
		\item Interface through Unity's \gls{Inspector}.
	\end{itemize}
\end{description}


%\subsection{Augmented Reality}
%\todo{Merge with background?}\\
%"Augmented reality (AR) is a live direct or indirect view of a physical, real-world environment whose elements are augmented (or supplemented) by computer-generated sensory input such as sound, video, graphics or GPS data. It is related to a more general concept called mediated reality, in which a view of reality is modified (possibly even diminished rather than augmented) by a computer. As a result, the technology functions by enhancing one's current perception of reality. By contrast, virtual reality replaces the real world with a simulated one. Augmentation is conventionally in real-time and in semantic context with environmental elements, such as sports scores on TV during a match. With the help of advanced AR technology (e.g. adding computer vision and object recognition) the information about the surrounding real world of the user becomes interactive and digitally manipulable. Artificial information about the environment and its objects can be overlaid on the real world."\cite{WikiAugmentedReality}
%\todo{Er denne seksjonen nødvendig lengre?}
%\todo{Kommenter eller skrive noe eget i tillegg, ikke ha en seksjon som bare er en copy paste uten å selv skrive noe.}

\section{Organization}
\label{sec:organization}

\subsection{Development Methodology}
\label{subsec:development_methodology}
In the project we had to have a framework for structuring and controlling the
development process. We have all been educated in different methodologies that
are used in development, but were not confident enough in any of them to fully
go for one of the standardized ways. We also thought that none of them fitted 
the way we wanted to develop the project. Therefore we decided to get inspiration
from different methodologies and plan for ourselves how we wanted to develop.

During the development all of us would be continuously working with the project and we would have
regular meetings. On a weekly basis we had a retrospective review on
the progress. A short description would be written from each person to
document the work that had been done, as well as work we planned to do the
following week. On a daily basis we updated each other on progress. At
the beginning of a work day we exchanged status on progress and which
problems we were having. By doing this we could stop each other if a problem
got too time consuming. If an issue arose we wrote it down on a blackboard to keep track of all issues.
We would also be held responsible towards each other so
that we could be assigned new tasks as we finished others. Milestones would be
important goals to reach and we were ready to assign more time if that
would be needed for reaching one.

In addition to the document for weekly review, we had a separate
document for decisions taken during the development process. These decisions
was noted with date and author. Decisions made by the employer was
also written in the same document. To keep track of time that we had used,
we decided to go with a time management program called Toggl\cite{Toggl}. With this tool we
could log who did what and how much time they spent doing it. Commit log from GIT would
also show what had been done with the code.


\subsection{Project Administration}
\label{subsec:project_administration}
Since the group members had additional subjects, we decided to make a work week
last from Monday through Thursday. Fridays would be left for the other
subjects. We also gave each other freedom to start the day and end the day as
we pleased as long as we worked a fair amount of time. If we needed time for
other efforts, i.e. lectures, this was also accepted. Core working hours has
mainly been between 10:00 and 14:00, but we have expected the minimum work per
day to exceed that.

Most of the time during development was spent together at a storage room we
inhabited at Mustad Industry Park. This made it easy to cooperate on difficult
problems and re-plan if progress didn't go as expected.

Daniel Granerud was chosen as a leader. This was mostly a formality, but he was
given the right to have the final word in conflicts if they occurred.
Delegation of work was done by the group as a whole. Tasks were given according
to skill or will as we saw this to be a fair way of doing it. As problems
arose and tasks were done, we talked to each other and gave insights into our
work. By doing this on purpose we would stay stronger if we needed to adapt,
i.e. if one of us would get sick.


\subsection{Report Organization}
\label{subsec:report_organization}

Our report is structured into 6 chapters where each chapters contains sections that are related to the encompassing chapter.

The chapters are as following:

\begin{enumerate}
	\item \textbf{Introduction.}
	\begin{itemize}
		\item[]
			The project and its participants, topic, context and
			goals.
	\end{itemize}

	\item \textbf{Design process.}
	\begin{itemize}
		\item[]
			How we designed and approached the project around our
			given requirements.
	\end{itemize}
	
	\item \textbf{Development process.}
	\begin{itemize}
		\item[]
			Description of the theoretical part of the project as
			well how we worked, what we did and with which tools.
	\end{itemize}
	
	\item \textbf{Product.}
	\begin{itemize}
		\item[]
			What we ended up with at the end of our development.
	\end{itemize}
	
	\item \textbf{Summary.}
	\begin{itemize}
		\item[]
			Round up the project and thesis with a conclusion and
			afterword.
	\end{itemize}
	
	\item \textbf{Appendices.}
	\begin{itemize}
		\item[]
			Bibliography, glossary, how we managed time and other
			neat tidbits.
	\end{itemize}
\end{enumerate}