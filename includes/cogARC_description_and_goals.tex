\inlineHeader{What is cogArc?}
cogARC is a small framework that uses Unity and the Vuforia library to let you make a selection of mini games where the user can move a set of frame markers in real life and see them interact on a screen that augments the game

\inlineHeader{Background}
This project is part of the PhD project Konstantinos (Costas) Boletsis. Our aim is that this project will help him in documenting a users cognitive changes by playing a assortment of m mini games where the player moves markers around in the real world and a augmented reality enabled device will show the player the goals for the game.

\inlineHeader{Goals}
Create a easy to use interface for Costas to both manipulate the already made mini games while also giving him the opportunity to create more mini games if wanted with limited need to program or change source files.

\inlineHeader{Augmented Reality}
''Augmented reality (AR) is a live direct or indirect view of a physical, real-world environment whose elements are augmented (or supplemented) by computer-generated sensory input such as sound, video, graphics or GPS data. It is related to a more general concept called mediated reality, in which a view of reality is modified (possibly even diminished rather than augmented) by a computer. As a result, the technology functions by enhancing one’s current perception of reality. By contrast, virtual reality replaces the real world with a simulated one. Augmentation is conventionally in real-time and in semantic context with environmental elements, such as sports scores on TV during a match. With the help of advanced AR technology (e.g. adding computer vision and object recognition) the information about the surrounding real world of the user becomes interactive and digitally manipulable. Artificial information about the environment and its objects can be overlaid on the real world.''\cite{WikiAugmentedReality}
