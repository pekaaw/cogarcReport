Following is a list of decisions and changes we as a group made during the development of this project in chronological order.

\begin{enumerate}
	\item 09.jan: As many as possible mini games should be created in a dynamic way.
	\item 09.jan: Graphics for the system does not have to be more advanced than what is shown in the design document.
	\item 09.jan: The cubes need only to have basic collision between others to achieve the functionality we need for the games.
	\item 09.jan: Our focus should be on making as many games as possible within the framework and not on making  a few perfect ones.
	\item 24.jan: We decided to shift our focus to be more on the framework itself and not the games within the framework.
	\item 30.jan: Costas wants us to implement a variation of the mobile app Wooords to replace a few of the other mini games.
	\item 31.jan: We decide to look into the implementation of Wooords at a later date in the development as it is a interesting game and will enhance the finished app since it is not similar to the other mini games.
	\item 10.feb: The group, along with Costas have decided to drop the memory mini game and the path mini game described in the design document.
	\item 17.feb: In the design document it implies that the loading screens should have all time leader board and a high score for the player. This is not consistent with the rest of the design document nor with our planning document and it was decided to instead have a high score for the current player on the loading screen instead.
	\item 06.mar: The design of the boxes is set in the inspector during the making of the game in Unity, the design can therefore not be changed during runtime.
	\item 10.mar: We decided to look into detached / integrated use of a AR library.
	\item 18.mar: Pictures that will be used as textures on the cubes must be in a folder called "/Resources/BoxDesign/" to be able to properly load them since we are not able to extract the path of where a texture is stored on disc.
	\item 27.mar: The levels for mini games will be procedural generated unless you want to make a Wooords mini game, where the levels will be read from file since these levels can not be easily randomly generated.
\end{enumerate}

\section{Something}
