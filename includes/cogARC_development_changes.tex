Following is a list of decisions and changes we as a group made during the development of this project in chronological order group by what content it affected.

Minigames
\begin{itemize}
	
	\item 09.jan: As many as possible mini games should be created in a dynamic way.
	\item 30.jan: Costas wants us to implement a variation of the mobile app Wooords to replace a few of the other mini games.
	\item 31.jan: We decide to look into the implementation of Wooords at a later date in the development as it is a interesting game and will enhance the finished app since it is not similar to the other mini games.
	\item 10.feb: The group, along with Costas have decided to drop the memory mini game and the path mini game described in the design document.
\end{itemize}
Design
\begin{itemize}
	
	\item 09.jan: Graphics for the system does not have to be more advanced than what is shown in the design document.
	\item 17.feb: In the design document it implies that the loading screens should have all time leader board and a high score for the player. This is not consistent with the rest of the design document nor with our planning document and it was decided to instead have a high score for the current player on the loading screen instead.
	\item 27.mar: The levels for mini games will be procedural generated unless you want to make a Wooords mini game, where the levels will be read from file since these levels can not be easily randomly generated.
\end{itemize}
Implementation
\begin{itemize}
	
	\item 09.jan: The cubes need only to have basic collision between others to achieve the functionality we need for the games.
	\item 09.jan: Our focus should be on making as many games as possible within the framework and not on making  a few perfect ones.
	\item 24.jan: We decided to shift our focus to be more on the framework itself and not the games within the framework.
	\item 06.mar: The design of the boxes is set in the inspector during the making of the game in Unity, the design can therefore not be changed during runtime.
	\item 10.mar: We decided to look into detached / integrated use of a AR library.
	\item 18.mar: Pictures that will be used as textures on the cubes must be in a folder called "/Resources/BoxDesign/" to be able to properly load them since we are not able to extract the path of where a texture is stored on disc.
	\item 29.apr: The restart button have been removed from the pause screen.
\end{itemize}
	
Following we have described the most influential or important changes during the development and implementation from the lists above.

\subsubsection{Framework instead of games}
In the beginning we wanted to make sure that we would be able to create all the games described in the design document, and after looking into how we should structure the programs and what scripts and the like could be shared between each of the mini games we realized that on the programming side of them there was little difference. We therefore decided upon focusing making a framework to work within instead of making each mini game special.

\subsubsection{Implementing a word game instead}
Shortly after beginning development and starting testing of the \gls{Frame Marker} we realized that the mini game \#5: Memory cubes as it was described in the design document was not implementable. Playing the game would make the player turn the cube upside down in the best case scenario or covering the frame marker and then turn the cube in the worst case. This caused us a lot of problems because of the way that Vuforia controls its frame markers. What Vuforia does when it is not tracking a frame marker or it loses the tracking is to deactivate the frame marker in the scene hierarchy which makes us unable to access any information contained within both the frame marker object itself and its children. This meant that we had no reliable way to detect if the player turned a cube to see the hidden mark under the cube or if it was simply re-detected by the tracker due either by the player obscuring the marker or something else obscuring the marker or the marker just being lost for a few frames. We brought this issue to our employer and after a short discussion we came to the conclusion to implement our own version of the mobile game Wooords and not make the mini game \#5: Memory cubes and also to not make mini game \#6: Path as the game was deemed % nu nu nu
Woords is a game where the player is presented with a jumble of words and a theme. The goal is to find as many or all the words in the level by combining the letters to make words.


\subsubsection{Dropping leader-board}
In the original design document we got there was mention of having the loading screen both instructions for the game and a global leader-board. Unity does not work well with net based activities 
so this feature was dropped.

\subsubsection{Putting textures in a special folder}
This decision is due to a limitation in Unity. When you place a object in the object field in the inspector it will give us a new instance of that object, in this case a texture. What it does not give us is where it is stored. This would not be a problem if Unity was able to serialize our data, but since we are storing our objects as a JSON object due to Unity not being able to serialize it we have to store it ourself and the name of the texture is the only data we can rely on. The name however does not include the path of where the object is stored, so when we are extracting the data back from JSON objects into textures we only have the name of the texture. So to avoid any confusion of duplicate names and avoiding searching for the texture we constrict the location of where it can be placed into a sub-folder in the Resources folder of Unity.

\subsubsection{Procedural generation over predetermined levels}
In the beginning of the development we were uncertain about whether we should create levels for the mini games with code or if it should be crafted by hand. We initially went for the possibility of both, however after discussing it with our employer we came to the conclusion that as many of the games as possible should have generated levels. The only mini game that is not suitable for random generation is mini game \#7: Wooord game, the Wooord game has a single file for each set up starting with the overall theme (example is: animal, this means that all the words are animal related), followed by a comma separated sequence of what words should be on the cubes and on the line under is a comma separated list of words that can be created with those words. To create more levels the file just have to continue with a new sequence of letters for the cubes to have and a new line of words that can be created.

\subsubsection{The restart button has been removed from the pause screen}
At first pressing certain buttons caused errors when Application.Load(levelNumber) was called. This is the function you call in Unity scrips when you want to switch scene or go from level 1 to level 2. Going to the next level in cogArc only uses this function when loading a new game (change of rules), when reloading current game or returning to main menu, not when going to next level in the same game. Sometimes when starting up the application or on application.load we would get an error saying that Vuforia could not initialize the trackers. This would happen at random, but at a very low rate. However when pressing the restart game button in the pause menu, it would happen every single time. We used the exact same code for the same button in the score-screen, but the error did not occur there. The error occurred during the application.load call and we had no way of closing in on the error further with the debugger. Because of this we decided to remove this button.\\
While going over some other code we realized that the error was caused by setting time.timescale to 0. Which we do when entering the pause menu. We thought this would be reset to 1 when loading a scene. In hindsight setting the timescale back to 1 is the obvious solution, but at the time there was nothing linking the cause to the error. The only error message we could get was that Vuforia could not initialize the trackers. After finding the real cause it was an easy fix and the button was put back in.  
