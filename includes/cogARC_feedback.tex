\begin{itemize}
	\item A friend from F\o{}rde tried the app on his Samsung S5. This caused his phone to restart. When he tried the app after it's restart, the phone only replied that the program had stopped.
	\item There has been reported that it was hard to install the program from
	our webpage. To do so, one need to accept programs from "unsafe" sources
	that are not on Google Play.
	\item People have asked for the app both for iOS and Windows Phone. We have
	tried to port cogARC to these platforms, but never succeeded. There may be
	some APIs and features that are not available in the Unity free edition.
	For Windows Phone the webcam is not available
	\cite{GettingStartedWindowsPhoneBuildUnity}, so that may be one of the
	problematic factors.
	\item Someone was observed to connect the frame markers by dragging
	on the screen from one of them to the other. This is probably since it is a
	new technology and may be most prominent the first time a person tries it.
	\item By trying on a smart phone someone was observed to hold it quite high
	in a sitting position. The reason was to see all the frame markers at the 
	same time. It was an uncomfortable position and the player started to 
	recognize it in their back. This was played without cubes, but with flat
	markers on the table. It may not be a problem when playing it as the
	employer have described -- with cubes in a sitting position with a tablet
	in front(bigger screen).
	\vspace{8pt}
	\item \nameref{game:pattern_memory}:
		\vspace{-8pt}
		\begin{itemize}
			\item Friend did not see the pattern in the corner when it
			started. A possible solution could be to put it at the middle of
			the screen to make it easier to see.
			\item Someone experienced that the game said the solution was both
			right and wrong at the same time. Algorithm optimization needed.
		\end{itemize}
	\item \hyperref[fig:pair_games]{Pair games}: Some got paired up without the user recognizing it. This
	we have seen before. A possible solution could be to show how many
	pairs that are left. On \nameref{game:shape_match}, it could e.g be 
	pictures of those that are left.
	\item \hyperref[subsec:vuforia]{Vuforia}: Marker recognition not good
	enough. Problem with seeing	markers are often problematic.
	\item General: Player felt frustrated and confused when not understanding
	the games. This changed upon completion of a level, where instead there 
	were rewarding feelings of joy and victory (\nameref{game:pattern_memory}).
\end{itemize}