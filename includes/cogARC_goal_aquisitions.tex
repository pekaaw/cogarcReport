\begin{description}
	\item[Result goals]\ 
	\begin{itemize}
		\item 7 games.
		\item Adaptable framework.
		\item Viable bachelor thesis.
		\item Combine physical cubes and digital structures.
	\end{itemize}
\end{description}

These are the result goals we started out with. During the process some of them
were altered a little. Some of the game id\'eas were dropped due to technical
issues. Some were decided to be abandoned in conversation with our employer.
They were simply not interesting or necessary enough. This selection process
was done over time in the early phases of the project. In the end, we finished
every game that were planned. One of the games we dropped were implemented
close up to the deadline, but finalization with proper testing were not
prioritized. This shows that the framework is adaptable and it does not take
very long time to expand on it to create new simple game types.

The framework we made is flexible beyond what was required to make the
requested games.
The rules of games can easily be changed in Unity. 
There are several options for changing the difficulty of games, such as setting time limits, setting number of cubes used to form solutions and setting number of levels.
For the \nameref{game:wo0ord_game}, the file containing the tasks can easily be changed. 
The game description can be changed and the score system can to some degree be tweaked in the inspector.
Adding a new game in main menu may not be straight forward, but there is not
much to it. It is as described in \autoref{sec:adding_new_game}
\nameref{sec:adding_new_game}. An interesting feature we also have is to create
a serie where all the games are put after each other. As cogARC are built at
the moment, a round will take about 20-30 minutes to play.

From our perspective it may be difficult to be objective about whether our 
bachelor thesis is viable or not, but we think we have completed the task well.
The \nameref{game:wo0ord_game} and the inspector-framework
(\ref{sec:editor_data}) were big additions to the initial assignment.
Time-wise we have logged a little less than we might have expected, but we
think it is not so uncommon.

When it comes to the last result goal, we are quite satisfied.
We turned the volatile environment seen by the cameras of the real world
into fairly stable arrays and lists and tested them against both preset 
and randomly created states. Some buggy behavior occurs, but most of this
is because the images from the camera are too bad for the trackers or because
of Vuforias tracking algorithms and therefore such that we have little or no 
control over the matter.

\clearpage

\begin{description}
	\item[Effect goals]\ 
	\begin{itemize}
		\item Support research.
		\item Experience with a game engine.
		\item Experience with a longer development process.
		\item Experience with digital space linked to reality.
	\end{itemize}
\end{description}

We cannot really say at this point how well our product will support research,
but our employer stated that he was pleased with the product and we were able
to meet most of the requirements for logging events. Hopefully the work we have
done can at least be a little useful.

During the project we have gained a lot of experience with the things we aimed
for.
It was the first project we did in Unity, so both the game engine, coding
language and the tools we have used have been new to us.
In the beginning we had a lot of trouble figuring out how to structure, set up
the scene hierarchy, where to put scripts and how to make scripts communicate.
As we implemented new features we went over the structure several times, improving it a little each time. 
The Unity project still shows signs of structural weaknesses.
We chose not to remove all of them because it would mean a lot of refactoring work. 
We believe that in future projects in Unity and similar, we will be able to set much better structures because of the experience we gained here.

When it comes to the field of AR, we have not been exploring it as much as
we could have. By using Vuforia we got a good introduction with AR. We could
have experimented a lot with it, but even though we studied some possibilities,
we chosed to focus on finishing the mini games and creating the framework
instead.