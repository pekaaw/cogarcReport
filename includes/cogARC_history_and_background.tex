\subsubsection{Augmented Reality}

\gls{Augmented Reality} is a quite new discipline in computer science. It uses
a real-world environment and augments its perspective with adding digital
content. One of the biggest contributors in the field is Steve Mann, who has been
working with wearable computing and augmented reality systems since the 1980's.

The \gls{AR} term itself was coined at Boeing\cite{boeingAR} in 1990 where the
technology was
proposed to replace a lot of large plywood boards with schematics for planes.
They proposed a kind of eyewear that could project the schematics on reusable
boards and with this make reconfiguring a whole lot easier during the
manufacturing process.

Even though most of the AR technology of today happens through screens (i. e. like
mobile phones, Google glasses\cite{GoogleGlasses} and Meta SpaceGlasses\cite{MetaSpaceGlasses}),
the id\'{e}a of projection is still living. castAR is an example of this 
technology where digital content is cast on a reflective material.
The technology has been around for some decades and is being present in several
forms, but the technology has needed a long time to mature. It is only recently
that algorithms and hardware has started to be good enough for everyday people
to take interest, and because of this we will still consider it as a quite new 
discipline in computer science.

\subsubsection{Serious Games}

'A serious game or applied game is a game designed for a primary purpose other than pure entertainment.'\cite{wiki:SeriousGames}

Serious Games differ from other games in that its primary aim is not solely to entertain, but to educate or inform the player.
This can streach from simulations used by the military to simulate combat, to educational games used by schools to teach kids math.

Serious games have their roots in education as alternative ways to teach.
Serious games is not exclusive to computer games as serious games have traditional been board games or card games.
These kind of games became popular in the 1960s and 1970s, but in later years computer based games have become more 
popular. The reason for this is likely to be that handheld computers became more available and more developers and companies have created games and educational tools for them.

In more recent years serious games have evolved outside the educational market as computers have become powerful enough to simulate real life or real life situations.
Particularly usage in the military and advertisement sector have expanded the field of serious games to expand into a variety of areas.
Some examples are newsgames that gives the user a new experience of reading the news, mobile phone apps that give the player points each time they do a specific activity
such as visit a new place or complete a training exercise, or games aimed for improving knowledge about health issues or improving ones health.

\subsubsection{Mild cognitive impairment}

Mild cognitive impairment(MCI) is the noticeable decline of cognitive functions beyond what is expected based on age and education.
MCI have often been found to be a transitional stage to Alzheimer's disease. \cite{MCI_alzOrg}

MCI is classified in two ways based on the thinking skills that is affected:
amnestic MCI is the decline that is most prominent in primary memory, signs being unable to recall or forget important information that
would previously be easy to recall such as conversations.

The other classification is nonamnestic MCI which affects other parts of the memory such as being able to complete
a complex task or make a decision quickly.

There are currently no tests or procedures to find or prove MCI conclusively in a person and a treatment for the decline has yet to be found.
