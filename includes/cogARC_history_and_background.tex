\gls{Augmented Reality} is a quite new discipline in computer science. It uses
a real-world environment and augments its perspective with adding digital
content. One of the biggest contributors in the field is Steve Mann, who has been
working with wearable computing and augmented reality systems since the 1980's.

The \gls{AR} term itself was coined at Boeing\cite{boeingAR} in 1990 where the
technology was
proposed to replace a lot of large plywood boards with schematics for planes.
They proposed a kind of eyewear that could project the schematics on reusable
boards and with this make reconfiguring a whole lot easier during the
manufacturing process.

Even though most of the AR technology of today happens through screens (like
mobile phones and Google glasses / Meta SpaceGlasses\cite{MetaSpaceGlasses}),
the id\'{e}a of projection is still living. castAR is an example of this 
technology where digital content is cast on a reflective material.
The technology has been around for some decades and is being present in several
forms, but the technology has needed a long time to mature. It is only recently
that algorithms and hardware has started to be good enough for everyday people
to take interest, and because of this we will still consider it as a quite new 
discipline in computer science.

The background for the project is that our employer Costas Boletsis is doing
a PhD and wanted a set of games that tests the players cognitive condition
by completing defined tasks. To do this he wanted to use a device that
implements augmented reality on a set of physical cubes that the user can
interact with.
When we decided we wanted to go for the task, we were expecting to use the
Meta SpaceGlasses. They were expected to be released in December 2013, but
unfortunately the release of these has been pushed forward to July
2014\cite{MetaSpaceGlasses}.
Because of this we have been developing for the Android platform. During the
development process we have tested on our own phones, but towards the end
of development, as well as during testing, we have also tested on a tablet.


We got a set of game specifications that our employer wanted us to implement.
Many of the games had similar features, so we decided to do the project a
little more useful.
Instead of making several independent games, we decided to create a kind of 
platform or environment. In this piece of software we have tried to make it
easy to create and modify small games with a certain set of features. Our
hope is that our employer and others can use it as a tool to perform research
and medical analysis. The environment will contain predefined objects that
can be manipulated and put together to achieve different goals for measuring
a number of cognitive functions.
