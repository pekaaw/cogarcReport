cogARC is a bachelor project under Gj\o vik University College. We have been
working with \gls{Augmented Reality} to create a tool for cognitive research.
The tool has form as a game with logging functionality that can be used by
researchers. cogARC can typically be used by patients that has suffered stroke
or people with declining cognitive functionality. Doctors, medical staff and/or
researchers can use logging information to get an overview over the cognitiv
situation for the patient and can be enabled to track improvements or
deterioration.

\cite{GenVirtual}



% What is the setting of the problem? This is, in other words, the background. In some cases, this may be implicit, and in some cases, merged with the motivation below.
% What exactly is the problem you are trying to solve? This is the problem statement.
% Why is the problem important to solve? This is the motivation. In some cases, it may be implicit in the background, or the problem statement itself.
% Is the problem still unsolved? The constitutes the statement of past/related work crisply.
% Why is the problem difficult to solve? This is the statement of challenges. In some cases, it may be implicit in the problem statement. In others, you may have to say explicitly as to why the problem is worthy of a BTech/MTech/PhD, or a semester project, as the case may be.
% How have you solved the problem? Here you state the essence of your approach. This is of course expanded upon later, but it must be stated explicitly here.
% What are the conditions under which your solution is applicable? This is a statement of assumptions.
% What are the main results? You have to present the main summary of the results here.
% What is the summary of your contributions? This in some cases may be implicit in the rest of the introduction. Sometimes it helps to state contributions explicitly.
% How is the rest of the report organized? Here you include a paragraph on the flow of ideas in the rest of the report. For any report beyond 4-5 pages, this is a must.