This report covers the bachelor project named cogARC under Gj\o vik
University College, Spring 2014. We have been working with \gls{Augmented
Reality} to create a tool for cognitive research. The tool has form as software 
with minigames that has logging functionality that can be used by researchers. 
The cogARC software can typically be used by patients that has suffered stroke
or people with declining cognitive functionality. Doctors, medical staff and/or
researchers can use logging information to get an overview over the cognitiv
situation for the patient and can be enabled to track improvements or
deterioration.


Thanks goes to Costas Boletsis for being a great employer and giving us good feedback
throughout the project. He has been availible and able to give answers when we
needed it.

