\subsection{Implementing logging}

Due to some miscommunication with our employer we thought we weren't supposed to implement logging as we did not have any means of logging the movements of hands. 
We thought this also meant that we would then drop all logging in the game (user high-score list not withstanding). 
But about two weeks before the due date it was brought to our attention that logging was essential for our employer to have. 
So with very little time left of the project we had to add logging of the user interaction and store the information in a way that was both accessible and useful to our employer.
To fulfill these requirements we had the help of our supervisor Simon McCallum to set up a web server where we could post our data to. 
He set up two database tables to which our application sends data, one for keeping the scores as an online high-score list and one for the events our employer needs for his research.\\
This part of the application was implemented based on verbal requests and some Skype messages with our employer and supervisor. 
The requirements were developed as we went along and were added and removed as we implemented and discussed. 

The following is a somewhat detailed description of what we implemented and our interpretation of the specification.
\paragraph{Logging:High-Score}
On completing all the levels of the selected mini-game the application will send final score, username (chosen by the user in the game, we use a special identifier if the username is not set) and the ID number of the mini-game selected to the online high-score list. The mini-game ID number is a number from 1-6 for the individual mini-games and a higher number if more mini-games are played in succession.
\paragraph{Logging:Events}
During each mini-games, events are created and stored in an array. After each level of of each mini-game, events are sent to the server.
Events are made when:
\begin{enumerate}
	\item the player completes a goal in a mini-game, for example matching a pair.
	\item the player makes a mistake, for example setting up a wrong formation of 3x3 cubes in \nameref{game:pattern_memory}.
	\item in the \nameref{game:wo0ord_game} mini-game only, the player sets up a word that was correct, but has already been checked and a different word has been confirmed in the meantime.
\end{enumerate}
The event-data consists of username, current score, mini-game ID, current level in current mini-game, event type, combined data of the cubes involved in the event and the time of the event. 
Event type is an integer 1-3 referring to how the event was made, matching the list above. 
The combined data of the cubes involved will for example spell out the word in \nameref{game:wo0ord_game} or the bit-sequence of colored and uncolored cubes in \nameref{game:pattern_memory}.\\

\paragraph{Implementing logging}
For most of the games it was easy to implement, but for the word-game (\ref{game:wo0ord_game}) our employer and supervisor wanted some extra data that was not possible to get the way the code was written. 
They wanted to log events when a player after completing a registered word forms another registered word  then tries to form first word again. 
Problem was that for all the games to use the same program flow and scripts we made a system were completed goals no longer exists in memory. 
\begin{enumerate}
	\item LevelCreator fills up a goal-state-array that may consist of several words or tasks.
	\item When a goal is completed, a correct word is formed, a pair is matched or the correct 3x3 grid has been made, this part of the goal-state-array is removed.
	\item When the goal-state-array is empty (or the time is up when there is a time limit) the array is cleared and next level is loaded.
\end{enumerate}
Problem was that we had no way of testing if the newly formed word is one that has been done before as the word had then been removed from the list of words we were looking for. 
We solved this by putting the completed words into another list and duplicate most of the test-function to test against this new list after testing for correct words.\\ 
It went surprisingly fast to implement, taking only a few days to have all games sending the correct data to the server with data our employer can use.


\subsection{Using player preferences to store data}
To store data that we need later such as username or scores from minigames we use Unitys built in methods for storing data.
Using the built in methods lets us have Unity take care of storing the values correctly on any platform cogArc would be installed on without having to worry about 
storing and retrieving the values correctly ourself.
Unity supports storing of simple data such as strings or integer values paired with a key.
To store more complex data such as the scores from each minigame we use a third party library called PlayerPrefsX\cite{PlayerPrefsX}.
Using PlayerPrefsX we can store the scores as arrays that are much easier to retrieve than having one value paired with one key for each score.
Not using PlayerPrefsX would force us to store ten different key-value pairs for each list of scores we wanted to save.
Storing and retrieving the scores for a game is done with a compound key made from username + the name of the game.
For example if the username is set as "John" and the game is "Total Sum" the key used to retrieve and store values would be "JohnTotalSum".

Splitting the highscores on both names and games allows us to easily create highscore lists without having to worry about it interfering with
a highscore of another game or if a device is shared it is easy for the user to have individual scores by changing their username.
