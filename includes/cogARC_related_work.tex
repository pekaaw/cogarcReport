\gls{AR} is a technology that is starting to get popular. Because of this there
are not that many contributions to the field that is too similar to our project.
At the University of S\~{a}n Paulo, some academics have created a musical AR game
that has noticeable similarities with our work, they have called it GenVirtual
\cite{GenVirtual}. 

There exists AR games both for entertainment and for more purposeful
measures\cite{tan2010augmented}. In 2000, ARQuake stood out as the first fully
working outdoor AR game. It needed a lot of equipment attached to the body, so
it was never commercialized. Anyway, it generated a substantial interest in the
AR community. From 2005 and onwards more AR games have appeared. As
equipment have become cheaper, lighter and less space consuming, it has opened a whole new world of possibilities. 
Smartphones and tablets have brought forth
small devices with increasing processing power. With freely available libraries
as \gls{Vuforia}, many new applications are expected.
