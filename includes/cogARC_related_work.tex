\gls{AR} is a technology that is starting to get popular. Because of this there
is not that many contributions to the field that is too similar to our project.
At the University of S\~{a}n Paulo, some academics has created a musical AR game
that has clear similarities with us, they have called it GenVirtual
\cite{GenVirtual}. 

There exists AR games both for entertainment and for more purposeful
measures\cite{tan2010augmented}. In 2000, ARQuake stood out as the first fully
working outdoor AR game. It needed a lot of equipment attached to the body, so
it was never commercialized. Anyway, it generated a substantial interest in the
AR community. From 2005 and onward there have appeared more AR games. As
equipment have been cheaper, lighter and less space consuming, there has opened
up a whole new world of possibilites. Smartphones and tablets has brought forth
small devices with increasing processing power. With freely avalible libraries
as \gls{Vuforia}, many new applications is expected.
