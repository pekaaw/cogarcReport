\begin
\chapter{UnityScript}
In Unity a user can use several different languages to script their games. Boo, C#, JavaScript and Shader (used for writing shaders).
While Boo and C# use the standard implementation and is easily referenced the JavaScript used in Unity is not.
% Help Hvordan setter jeg inn http://wiki.unity3d.com/index.php/UnityScript_versus_JavaScript ?
% http://en.wikipedia.org/wiki/ECMAScript Denne også
Unitys JavaScript is quite different in many ways from the more standardly used JavaScript that is based on ECMAScript which is a standardized scripting
language standardized by Ecma International.

Some of the major differences between Unitys JavaScript (henceforth named UnityScript) and ECMASCripts JavaScript (henceforth refered to as JavaScript):
\texttt{itemize}
\begin{itemize}
	\item Classes.
	\item Different privacy.
	\item Not JavaScript library compatible.
\end{itemize}

\subsection {Classes}
JavaScript have no concept of classes as it is a prototypical language, meaning it will have the properties that is assigned to it, but is not bound to them.
Properties can be added or deleted at will in runtime, this is true for both members of the object and functions. UnityScript is much more object oriented with strongly typed and defined classes that cannot be changed in runtime (the exception to this rule is reflection).
This makes it difficult to work with UnityScript if you have little experience using it, or have yet to find out that the difference between JavaScriptand UnityScript



\subsection {Different privacy}
UnityScript, being structured like other object oriented languages have the usual levels of privacy of members within a class being markedas either Private, Protected or Public. JavaScript have per function privacy. Meaning a variable declared in a function is available in the entire function but not
outside of the function.

\subsection {Not JavaScript library compatible}
UnityScript being so different from JavaScript makes other pre-made JavaScript library non-compatible with UnityScript, which also makes it really difficult to find reference help. One half of the built in library in Unitys Monodevelop is Unitys own library and the other half is made using what seems to be .NET framework from MicroSoft

% I guess end subsection here
