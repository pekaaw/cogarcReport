In Unity a user can use several different languages to script their games. Boo, C\#, JavaScript and Shader (used for writing shaders).
While Boo and C\#  use the standard implementation and is easily referenced the JavaScript used in Unity is not.\cite{WikiScriptVSScript}

Unitys JavaScript is quite different in many ways from the more standardly used JavaScript that is based on ECMAScript\cite{ECMAscipt} which is a standardized scripting
language standardized by Ecma International.

Some of the major and notable differences between Unitys JavaScript (henceforth named UnityScript) and ECMASCripts JavaScript (henceforth referred to as JavaScript):

\begin{itemize}
	\item Classes.
	\item Different privacy.
	\item Not JavaScript library compatible.
\end{itemize}

\subsection {Classes}
JavaScript have no concept of classes as it is a prototypical language, meaning it will have the properties that is assigned to it, but is not bound to them.
Properties can be added or deleted at will in runtime, this is true for both members of the object and functions. 
UnityScript is much more object oriented with strongly typed and defined classes that cannot be changed in runtime (the exception to this rule is reflection).
This makes it difficult to work with UnityScript if you have little experience using it, or have yet to find out that the difference between JavaScriptand UnityScript.
UnityScript have also removed the prototypical properties of JavaScript by implementing it as a class based language. 
Meaning objects can not get additional functionality or be completely redefined at runtime. 
In UnityScript each .js file is implemented as a class by default. 
So when a developer creates a new .js file the Unity compiler and serialization system will add a class definition around the data within the file by default, so to call a function within a .js file you write "filename".function.
This however is not documented and gave us problems in the starting phase when we were learning UnityScript as there was no obvious way to get data or functions from other files.

\subsection {Different privacy}
UnityScript, being structured like other object oriented languages have the usual levels of privacy of members within a class being marked as either Private, Protected or Public. 
JavaScript have per function privacy. 
Meaning a variable declared in a function is available in the entire function but not outside of the function.
But sometimes the compiler gets this confused or ModoDevelop get this confused as we have had problems with a variable being accessible in one part of the function but not in other parts.
This has been the case with member variables of the script and variables in the function not being available, this led to having some times having to create temporary variables to store the original variable to be able to use it in the fucntion.
%
%\subsection {var as a keyword is required}
%If you do not use the keyword var in JavaScript the variable becomes a global variable as the var keyword is used to denote scoping. In UnityScript the var keyword is required when creating a variable to make sure the scoping is always the intended scope.

\subsection {Not JavaScript library compatible}
UnityScript being so different from JavaScript makes other premade JavaScript library non-compatible with UnityScript, which also makes it really difficult to find reference help. 
One half of the built in library in Unitys Monodevelop is Unitys own library and the other half is made using the .NET framework from MicroSoft.
This was a small problem when we looked into using {JSON} objects in our program and wanted to use a library.
Since {JSON} is a part of the JavaScript language and specification we thought that it was also implemented in UnityScript.
This was not the case, but we are not the first to notice this and there are now several {JSON}-parser libraries available created for Unity in both C\# and in UnityScript.
One such library is Boomlagoon\cite{Boomlagoon.JSON} that we ended up using for our project.
