\subsection{Skipping levels}
During a game there might come up situations where you just want to quit. At
first we did not have this possibility, but during the testing phase we figured
out that this was something that had to be included. Particularly during the
Pattern Memory game (\ref{game:pattern_memory}) we think this is necessary. In
that game you will need to remember a pattern. If you fail to do so, you may
spend much time trying and failing to get the right pattern and frustration can
raise to an unfortunate level. 

\subsection{Total Sum Hint}
In the Total Sum game (\ref{game:total_sum}) it can be quite frustrating to add up the numbers in your head before hand, especially if the number of cubes you have to add up is more than three. When the player puts together two or more cubes in the specified way, (standard is vertical, but in the unity inspector the rule can be changed to the horizontal version, "human readable" ) a text will appear on the top box or last box in the line showing how much the current sum is. The hint will only appear on one tower / word.

\subsection{AR Implementation}
AR implantation will be explained in \ref{subsec:framemarker_model}.
When we were implementing Vuforia we focused on usability rather than aesthetics. The twitching, incorrect rotation and disappearing of the cubes when using the standard way give the player more nuanced feedback about the conditions for the tracking and understanding for how it works. Even though we were having tracking problems it was better to not make changes. There were other ways that could make the cubes easier to get into and keep in view, which would make some games easier to do if playing in bad conditions. but also in another way make the game harder because cubes would get stuck on screen or be in wrong positions confusing the player.



\todo{What have we thought through about usability? What can we make up here?}