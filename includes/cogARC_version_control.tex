We have been using GIT for version control in this project. This is a version
control system we had some prior experience with, so we wanted to try it for 
this project as well. 

GIT works best with plain text files. With Unity there is quite many binary
files as well. At first we were not entirely sure how much of the files that
were binary and if GIT could work for us. After some googling we found that 
this was possible. On Unity's webpage they instruct on how to use the version
control system called Subversion \cite{SubversionControl}. Together with some
help from stackoverflow.com\cite{gitVersionControl} we figured out how to set 
up the project with GIT.

Unity produces .meta-files for each file in the project. We do not know 
entirely how these works, so we met some problems that we suspected had its
origin in these files. Because of this we have tried both to have them in
the GIT repository and not. The solution to that specific problem was not
clear when we got it working, so we do not know whether the metafiles were
problematic or not. The post at stackoverflow explained that the .meta-files 
could be set to hidden. This implies that the files should not be part
of the GIT repository. Anyway, now thay are because it seemed to help with the
problem we had, and it has not given any other problems later.

Many of the pure binary files has not been necessary to upload. Those that
has, like images, has often been of a type that didn't need to be updated
often. This means that most of the content in the repository could be plain
text and therefore we have been able to use GIT without further hassle.

We have used GitHub \cite{GitHub} as our online repository server. By using
this service we could easily share the work we have done. That we could use
an external server like this has been important for us when choosing version
control tool. Especially in one occation this proved to be very smart. One of
us encountered a bug in Unity that were quite severe. The Unity3D program
crashed, and by some unknown reason, it deleted a lot of files. The files that
were removed were all of the project files + some other files in the parent
directory. Many totally unrelated files were deleted and 4 days work that was
not committed to the server were lost. But fortunately enough, he could
download the lastest version from GitHub and use the last days lessons to write
the lost changes a whole lot faster.

When we have been using GIT we have used the commandline tool and also the GUI
program for GIT called Sourcetree \cite{SourceTree}.

The workflow we have been using has been very simplistic. We are aware that
there are good workflows that we could have used, but we have choosen to only
have a master branch that everyone pushes to. During the project we were told
about a way that worked for another group were they used pull requests and in
this way added an extra layer of security of code integrity. We tried this out
for a short time, but since none of us had experience with this way of working,
we discared it after a relatively short time. We were working quite close with
many small and rapid changes that we shared with everyone. If another group
member should accept this every time, we think that a lot of unnecessary time
could have been wasted on this. The model is not bad, but we have come to the
conclusion that it didn't fit our way of working very well. If we had started 
to do it from the beginning, we could maybe have done it, but the effort may
still have not been worth it. Anyhow, we are glad to learn about other ways of
working that can be useful later.